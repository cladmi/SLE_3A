\documentclass[a4paper,10pt]{article}

\usepackage[french]{babel}
\usepackage[utf8]{inputenc}
\usepackage[T1]{fontenc}

\usepackage{amsmath}
\usepackage{amsfonts}
\usepackage{amssymb}

\usepackage{graphicx}
\usepackage{array}
\usepackage{pgf}
\usepackage{tikz}
\usepackage{listings}
\usepackage{eurosym}
\usepackage{wrapfig}

\newcommand{\unitnb}[2]{\unit{\nombre{#1}}{#2}}
\newcommand{\hexa}[1]{$\mathrm{\textbf{0x}}{#1}$}
\definecolor{vert}{rgb}{0.2,0.6,0.4} 
\definecolor{gris}{rgb}{0.4,0.4,0.4} 

\newcommand{\lettrine}[2]{\hspace{5mm}\LARGE{#1}\normalsize{#2}}

\newenvironment{changemargin}[2]{\begin{list}{}{
	\setlength{\topsep}{0pt}
	\setlength{\leftmargin}{0pt}
	\setlength{\rightmargin}{0pt}
	\setlength{\listparindent}{\parindent}
	\setlength{\itemindent}{\parindent}
	\setlength{\parsep}{0pt plus 1pt}
	\addtolength{\leftmargin}{#1}
	\addtolength{\rightmargin}{#2}
}\item }{\end{list}}

%--------------------EVITE LES ORPHELINS-----------
\widowpenalty=10000
\clubpenalty=10000
\raggedbottom
%--------------------------------------------------

\makeatletter
\@addtoreset{chapter}{part}
\makeatother

%\renewcommand{\FrenchLabelItem}{\textbullet}

\textwidth = 15cm
\hoffset = -1.54cm
\voffset = -1cm
\headsep = 0cm
\headheight = 0cm
\topmargin = 1cm
\textheight = 22cm

%% Package pour faire des liens dynamiques entre les pdf
\usepackage[pdftex,
		    bookmarks         = true,       % Signets
		    bookmarksnumbered = true,       % Signets numérotés
		    pdfpagemode       = None,       % Signets fermé à l'ouverture
		    pdfstartview      = FitH,       % La page prend toute la largeur
		    colorlinks        = true,       % Liens en couleur
		    urlcolor          = cyan,				% Couleur des liens externes
		    pdfborder         = {0 0 0}   	% Style de bordure
    		]{hyperref}

\hypersetup{
    pdfauthor   = {Adrien Oliva},
    pdftitle    = {Rapport de conférence},
    pdfsubject  = {Conférence technologique},
    pdfkeywords = {SLE conférence rapport avionique thales},
    pdfcreator  = {PDFLaTeX},
    pdfproducer = {PDFLaTeX}
}

\begin{document}

\begin{wrapfigure}{r}{2cm}
	\centering
	\includegraphics[width=40mm]{img/ensimag.png}
	\includegraphics[width=40mm]{img/thales_avionics.png}
\end{wrapfigure}

Adrien Oliva

\rule{5cm}{1pt}

\vspace{1cm}

\begin{center}
\Large
\textbf{Rapport de conférence technologique}
\normalsize
\end{center}

\begin{flushright}
Nicolas Blanpain et Pascal Fortin

24 septembre 2010
\end{flushright}

\vspace{10mm}

\rule{15cm}{0.5pt}

\vspace{2cm}

Cette conférence technologique, présentée par messieurs Blanpain et Fortin
de Thales, montre les enjeux de la programmation dans le domaine de 
l'avionique et les méthodes de programmation mise en \oe{}uvre au sein de 
leur société.

\section{Enjeux de l'avionique}

Quand on parle de logiciel embarqué au sein de l'avionique, il faut 
respecter un bon nombre de contraintes. En effet, dans un tel domaine, le
coût du bug est très important, tant au niveau matériel comme par exemple
le bug de la centrale inertielle du vol 501 d'Arianne 5 dont le coût est
estimé à 370 millions de dollars, qu'au niveau humain lorsqu'on parle de
système sur un avion de ligne de type airbus A350.

Les systèmes embarqués pour l'avionique sont à la convergence de plusieurs
problèmes techniques~: ce sont des logiciels \textbf{embarqués}, 
\textbf{temps réels}, et \textbf{critiques}. Ainsi, le logiciel doit 
satisfaire et répondre à des normes strictes. Dans le cas de l'avionique,
cette norme est la DO-178B qui contient 5 niveaux de criticité de A à E,
basé sur les conséquences que produisent un bugs dans le logiciel. Le 
niveau E est le plus bas en spécifiant que le problème est sans effet sur
la sécurité du vol, tandis que le niveau A implique un problème 
catastrophique conduisant presque inévitablement vers un crash de l'avion.

Pour de tels systèmes critiques, le coût de la ligne de code est le plus
important parmis toutes les lignes de codes écrites dans d'autres domaines.
En effet, il faut compter une ligne à l'heure pour un logiciel critique 
répondant aux normes DO-178B de niveau A. D'une manière générale, dans un
système critique, seul 20\% du code fait la fonctionnalité initiale~; le 
reste va gérer les cas particuliers et les effets de bord.

\section{Méthodes Agiles}

Les méthodes agiles consistent en des méthodes de développement logiciel
de type incrémentale, c'est-à-dire qu'à tout moment du développement, un
logiciel, même incomplet peut être livré à un client. Ce logiciel offre de
plus en plus de fonctionnalité au fur et à mesure des itérations.

\subsection{Historique}

Les méthodes agiles ont été introduites par le Manifeste Agile, 
officialisées en \oldstylenums{2001} par 17 personalités du génie logiciel.
Elles se basent sur 4 valeurs que sont \textbf{l'équipe}, qui est au centre
de la chaîne de développement, \textbf{l'application}, qui est finalement 
plus importante que la documentation associée, \textbf{la collaboration}, 
qui permet d'associer constamment le client à l'équipe de développement 
pour réellement répondre à ses attentes, et enfin, \textbf{l'acceptation au
changement}, qui permet une flexibilité dans l'application finale.

Même si l'ensemble des méthodes agiles fut officialisé en 
\oldstylenums{2001}, certaines d'entres elles ont été définies bien plus
tôt. Ainsi par exemple, la méthode SCRUM, utilisé par Thalès Aerospace à
Valence, est définie en \oldstylenums{1996}. Elle propose l'évolution du 
logiciel par fonctionnalitées. À chaque itération de la méthode, une 
nouvelle fonctionnalitée est implantée, testée, et documentée. Un autre 
exemple est la méthode XP pour eXtreme Programming, enoncé en 
\oldstylenums{1999}, qui préconnise des cycles de développement rapides et 
repose sur la programmation en binôme.

\subsection{Incrémentation}

Les méthodes agiles instaurées au sein de Thalès repose sur la combinaison
de la méthode SCRUM et de la méthode XP. Il s'agit donc de cycle de 
développement rapides basés sur les fonctionnalités du logiciel avec 
une programmation en binôme au sein d'une équipe soudée. Chaque cycle
d'incrémentation comprend les étapes suivantes~:

\begin{itemize}
\item définition de la nouvelle fonctionnalitée à implanter en accord et
avec le concours du client.
\item découpage et répartition de la fonctionnalitée aux binômes de 
programmation. Le découpage se fait en suivant la philosophie KISS (voir
la section~\ref{Kiss} pour plus de détail).
\item écriture des tests et vérification que ces test échouent sur la
version courante du logiciel.
\item implantation de la nouvelle fonctionnalitée et partage du code produit
avec toute l'équipe.
\item validation par, entre autre, les tests écrits en début de cycle. 
S'ajoute à cette validation les tests de non régression effectués chaque 
nuit sur la version courante.
\item écriture de la documention strictement nécessaire sur cette 
fonctionnalitée.
\item livraison de la nouvelle version au client
\end{itemize}

\subsection{Avantages face à un cycle en V}

Le cycle en V impose plusieurs prérequis. Tout d'abord, l'intégralité du
logiciel final doit être définit et spécifié dès le début. Ensuite, chaque
étape (spécification, architecture, conception, tests…) est confiée à des
personnes différentes ce qui implique forcément des problèmes de 
compréhension lié à la différence entre ce qu'on pense, ce qu'on écrit et
ce qu'on comprend. La méthode SCRUM évite ces inconvénients. À chaque cycle,
le client peu remodeler sa vision d'une fonctionnalitée~: la nouvelle 
itération va alors s'adapter à cette vision. Contrairement au modèle en V, 
les méthodes agiles offres plus de flexibilité. De plus, l'équipe de 
développement est associé dans son ensemble à toutes les étapes de la 
création du logiciel, allant de la spécification à la validation en passant
par l'implantation.

\section{Un mot sur la philosophie KISS}
\label{Kiss}

KISS est un acronyme pour \textit{Keep It Simple, Stupid}. Cette philosophie
impose par exemple de découper une fonction complexe en une multitude de 
fonction simple dont le fonctionnement est stupide. Comme le disait 
Léonard de Vinci~: ``La simplicité est la sophistication ultime''.

\section*{Sources}

Méthodes Agiles, article wikipedia, 
\href{http://fr.wikipedia.org/wiki/M%C3%A9thode_agile}
{http://fr.wikipedia.org/wiki/Méthode\_agile}

Méthodes Scrum, article wikipedia,
\href{http://fr.wikipedia.org/wiki/Scrum_(m%C3%A9thode)}
{http://fr.wikipedia.org/wiki/Scrum\_(méthode)}

Philosophie Kiss, wiki de la distribution Archlinux,
\href{http://wiki.archlinux.org/index.php/The_Arch_Way}{The Arch Way}

\end{document}
