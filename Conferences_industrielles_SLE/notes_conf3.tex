% definit le type de document et ses options
\documentclass[a4paper,12pt]{article}

% des paquetages indispensables, qui ajoutent des fonctionnalites
\usepackage[utf8]{inputenc}
\usepackage[T1]{fontenc}
\usepackage{amsmath,amssymb}
\usepackage{fullpage}
\usepackage{graphicx}
\usepackage{url}
\usepackage{xspace}
\usepackage{fancyvrb}
\usepackage[francais]{babel}

\begin{document}

\begin{center}
\Large{Systèmes Implantables et Applications pour la neurologie}\\
\vspace{0.4cm}
\normalsize{Conférence technologique de Guillaume CHARVET du CEA LETI.}\\
\end{center}

Chef de projet d'un développement d'un dispositif implantable actif au CEA / LETI / CLINATEC
\section{Presentation CEA et CLINATEC}
15000 employés avec un budget de 3Milliards d'euros
LETI de Grenoble : 1600 chercheurs : Micro Nano Technologiques \& Integration en Systèmes
 => MINATEC : 3500 personnes avec un budget de 320 Millions euros : CEA, CNRS, PHELMA ...

Mission de LETI :
Créer de l'innovation dans le besoin de l'industrie (fort lien avec l'industrie)

Axes de recherches
\begin{itemize}
\item Énergies \& Environnement
\item Photoniques \& Multimédia
\item Biologie \& Santé 
\item ...
\end{itemize}

Biologie :
\begin{itemize}
\item In vivo electonics
\item bio sytem on chip
\item preclinical molecular imaging 
\item X rayonnement
\end{itemize}

Minatec - Nanobio : nouveau bâtiment spécialisé.

\subsection{Clinatec} 
DEF :  nouveau centre de recherche médical dédié pour les micro nano technologies pour la santé (Alim-Louis Bernadid -- Maladie de Parkinson : implants médicaux.
Partenariat avec le CHU.
Objectif : regrouper les micro nano tech, traitement signal, chirurgie, médecines, bio-medicales, experimentation sur animaux ...
POURQUOI : Besoin des outils du domaines, pour plusieurs pathologies.
5 en 2008 (financé par le CEA). Batiment livré en 2011. Technologie dernière génération en médical, pseudo hôpital avec bloc opératoire.
équipements : MEG (mesure EM du cerveau), bloc chirurgicale...


\section{Etat de l'art des implants médicaux du commerce}
3 grandes familles des dispositifs médicaux : DM(dispositif médical), DMIA(Implantable Actif), DMDIV (DM de diagnostic in vitro)

Risque augmente selon les classes de DM (du médicament à la prothèse active, I, IIA, IIB, III (DMIA inclues))
ex: risque mécanique avec un implant : rupture de l'implant, lésion sur le corps ...

DMIA : 
\begin{itemize}
\item stimulateurs cardiaques et défibrillateurs
\item stimulateurs neurologiques
\item stimulateurs musculaires
\item ..
\end{itemize}
ex : pacemaker : stimuler le coeur par des impulsions : le bloc est éloigne du coeur, il transmet les impulsions par des sondes.
neurostimulation : auditif, visuel vésicaux ...
industriels : Medtronic, St Jude Medical, Cyberonics, boston scientific, ..
>15 \% par an

4 grandes classes d'appareil : pour parkinson(deep brain stimulation), épilepsie(vagus nerve ), douleurs, tympans
Stimulation cérébrale profonde sonde dans le cerveau : opération sous conscience du patient (le patient doit faire certaine activités : risques cassures d'électrodes).


ex : Épilepsie : en recherche
ex implant cochléaires pour des systèmes auditifs détérioré : micro -> antenne (RFID) -> électrodes dans la cochée pour refaire le son en signaux électriques directement dans le nerf pour le cerveau. 
ex vision artificiel : en projet ; caméra sur lunette par ex, et relié à l'oeil par un implant sur l'oeil (intérieur), batterie par la lunette => distingue ombres (ligne blanche)).

Cycle de dvlp d'un DMIA : Recherche fondamentale --> Recherche appliqué --> Prototype --> Dossier de conception (études cliniques) (critique pour le projet) --> CE marché .. --> Comcommercialisation --> Mise en service --> Utilisation  ===> 10 ans
Valider par l'AFFSPAS (association française de sécurité sanitaires des produits de santé)

Limitations des implants médicaux actuels :
- stimulation électriques uniquement
- macro électrodes (mm et non micro)
- nombre limité d'électrodes

Besoin :
pour les diagnostiques (épilepsie)
bouclage ( mesure/stimulation) pour épilepsie
bouclage pour DBS (deep brain stimulation)
mdeure multivoie pour appl BCI (cerveau - machine

\section{Recherche \& Développement technologiques pour les neurosciences}



pose des sondes directement dans les réseaux de neurones par des matrices de micro-electrodes

MEA 
tissus neurales en contact avec les micro électrodes pour de meilleurs diagnostiques.
Applications : recherches en nanosciences, pharmacologie, nouveau implants thérapeutiques (DBS), BCI, prothèse rétinienne.


Vivo BioMEA : brain activity sur un rat
Épilepsie : ``trou noir'' sur les enfants de l'activité du cerveau (rat génétiquement modifié) : solution : stimulation dans le cortex pendant qq secondes et arrête immédiatement la crise. Résonance du cerveau.


ASIC
architecture ASIC  pour les implants neuronal

augmentation de l'offre et demande de permis


Algo sous matlab puis en vhdl pour des FPGA : contraintes electroniques  => méthode de gain de temps sur la cahine de production

Importance du sans-fils : danger du piratage mais permet des maj.

Futur : implants en 3D (couches en étages)

\section{Une thématique emergente : le BCI (Brain computer interface)}
Controle d'un robot par l'intermédiare du cerveau pour un handicapé (interaction avec l'envirronement par interaction cérébrale)

Patient actif.

Mesure EEG (EM par la peau) ...
EEG non invasif
ECoG inv (tableaux electrodes) (ex : surface du cerveau pointes dans le cerveau) (stable pour plus de 28j à demontrer pour l'homme, commerciale validée, plus pour la recherche)
Local F.P (tableaux multi - electrodes) (interieur ) ( qualité excellente, resolution,  pas stable, mort neuronale, difficilement applicable pour l'homme)
Single Unit A.P (micro electrodes)

EEG très courant mais le temps de pose très long car rajout de liquide pour aider les sondes à capter ( possibilité pour le grand publique pour EEG ou EMG, jeux videos ...)

ex de BCI non inv : P300 Speller => potentiel evoqué 300ms après une estimation visuel (flash) (actuellement 1-2 mots à la minute)
  utilisable pour communiquer pour les handicapés lourds

ex : BCI asynchrone : Brain Switch
  possibilité de transformer une intention de mouvement en commande (concentration forte requise)

Application pour un casque P300 rapide à installer

Pilotage d'un brat robotisé par un primate (méthode invasive) succès mais durée de vie de 1-2 semaines.

Connecteur sur la tete à brancher sur un pc(1cm3, risque d'infection et d'endommagement du cerveau). Le traitement d'information très lourd 

Electrodes dans le cerveau (3mm environ)

EEG : dépend d'un neuronne

ECOG : depend d'un reseau de neurone (duree de vie sup)
curseur 2D, fauteil roulant , exosquelette (grand nombre d'electrodes possibles)
\end{document}