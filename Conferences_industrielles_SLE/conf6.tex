% definit le type de document et ses options
\documentclass[a4paper,12pt]{article}

% des paquetages indispensables, qui ajoutent des fonctionnalites
\usepackage[utf8]{inputenc}
\usepackage[T1]{fontenc}
\usepackage{amsmath,amssymb}
\usepackage{fullpage}
\usepackage{graphicx}
\usepackage{url}
\usepackage{xspace}
\usepackage{fancyvrb}
\usepackage[francais]{babel}

\begin{document}

\begin{center}
\Large{Les langages de programmation pour les systèmes embarqués.}\\
\vspace{0.4cm}
\normalsize{Conférence technologique de E. JENN de THALES.}\\
\end{center}

\section{Introduction}
Thales : Internationale, nombreux domaines
=> calculateur avionique civile
Airlab : étude sur les ``technologies du logiciel'' :
Médiateur entre le monde de la recherche et le monde de l'entreprise
(commercialisable)

\section{La question est-elle pertinente ?}
Importance : encore nécéssaire malgré une partie automatisé.
Automatisation :
 - ``Sureté'' du langage
 - Lisibilité

Impact sur la productivité.
Certains cas évidents :
  ex : Assembleur vs Langage structuré
D'autres :
  Qualité 
  Nombre de lignes
  Coefficient d'effort

=> Nombreuses études plus ou moins partiales encore d'actualités.
\section{Quelques observations sur les langages ...}
Langage informatique : lié à la machine, compilable, optimisable .
==> Surtout : Déstinés à un individu (codeur, relecture ...)

Pptés :
- expressif abstrait
- description du problème
- cohérence
- compréhensible
- détéction d'erreurs
- le moins contraignant
- modularité

Pptés ambigues :
ex : abstraction
- serialisations
- gestion de mémoire
- calculs flottants

** Pb : Cycle de vie des objets :
pas toujours accessible
manuel ? : propriété donc gestion non locales donc pertes de
modularité, fuites mémoires.
** Solution : 

Comportement complexe : temps d'éxécution non linéaire

** Pb : Traiter des données ``réelles''
virgule fixe : manuel
** Sol :
Conv en entiers

Transivité de l'opérateur ``='' sujet à problème et autres ops.
Dépends du matériel et du compilateur.


1300 langages : problème ou diversité
selon : framework, technologie, principe, industriel amène chaucun un
langage. Curiosité - ``mode'' => diffusion rapide des concepts



******


Le besoin d'un langage dépend du domaine (marché, technologies, champs
d'utilisation, pratiques )

Evolution des langages inéxorable mais aussi de vieux langages restent
tels que l'ADA et le FORTRAN
\section{Les systèmes avioniques embarqués }
Certification, sureté, temps, outillagen volume/poids/dissipation,
environnement, maintenance

Développement : 
Donner confience (avoir confiance) 
Contrainte de certification

Impact : 
- intérpréter/décliner obj de la DO, (code mort ? éxec ?)
- atteindre les obj de la DO
\section{Contraintes sur le choix du langage}
Contraintes dimensions 
contraintes mécaniques
contraintes tenue éléctronique
contraitens environnemental (important : dissipation)

sols : code correcteur, protection du cache contre rayonnement (tous
les caches car sinon on peut douter de l'intégrité des données)

Problème de ressourses :
 capacités et besoins croissants
 besoin d'évolutions

** Contraintes Architecture.
** Contraintes de dvlpt (reduire la prob de faute...) : favoriser
modularité et testabilité => ex : C et ADA

Rêgle de codage (ex : allocation dynamique et récursion interdites)
pour la certification

COntraintes ``runtime''
embarquement des librairies, API ...
Le code source ET le code final doivent être vérifié (certains
morceaux du code peuvent être généré étrangement par le compilateur).
Difficulté d'utilisation du java.

Contraintes temporelles :
temps réel

****** 15-20 min

Nombreuses contraintes ...
Langages différents -- besoins différents

Le C est l'un des langages les plus fiables mais le plus utilisés dans
l'embarqués.

Ccl :
Identifier les sources de changement,
Anticiper les changements,
Intégrer les changements utiles (tendance)
Eviter les changements inutiles ou dangereux 

Résistance à l'évolution : coût industriel, culturelle, ``pedigree''
du langage

Exemple : KCG

Passage du C++ à l'OCml

\section{Changements en cours et à venir}

Support multi-processus ...
support conception objet
contractualisation 
reconfiguration (futur)
distribution des traitements  (futur)

Moyens de com : A429 (primaire) -> AFDX (complexe) l'A... (bus de com
direct entre calculateur et actionneur)
DO178C (2011)
=> prise en compte des ``nouveaux'' langages, polymorphisme, gestion
dynamique de la mémoire, virtualisation

méthode formelle (modèle checking, interprétation abstraite)

Conclusion :
changement de langage par nécéssité technique et économique
composantes du choix : tech, culturelle, historique, reglementaire
tendance : reduire l'activité de codagepour l'application embarqué
\end{document}